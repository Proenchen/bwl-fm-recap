\section{Kurzfristfinanzierung und Working Capital Management}

\textbf{Motivation}: Wahrung des finanziellen Gleichgewichts erfordert
\begin{itemize}
	\item Detaillierte Planung zukünftiger Ein- und Auszahlungen, um den Kapitalbedarf rechtzeitig zu identifizieren
	\item Bestimmung der vorzuhaltenden Liquiditätsreserven (Cash Management) und die Messung von Liquidität
	\item Verhindern von absehbaren oder akuten Liquiditätsengpässen (Working Capital Management, Kurzfristfinanzierung)
\end{itemize}
\bigskip
\textbf{Was ist Cash bzw. Liquidität?}
\begin{itemize}
	\item \textbf{Zahlungsmittel}: Kassenbestand, Kredite und Schecks
	\item \textbf{Zahlungsmitteläquivalente}: Kurzfristige, sehr liquide Geldanlagen wie z.B. Schatzbriefe oder Geldmarktfonds $\rightarrow$ leicht veräußerbar, geringe Wertänderungsrisiken
\end{itemize}
\bigskip
\textbf{Motive und Determinanten der Liquiditätshaltung}:
\begin{itemize}
	\item \textbf{Motive}: 
\end{itemize}