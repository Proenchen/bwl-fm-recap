\section{Kurzfristfinanzierung und Working Capital Management}

\textbf{Motivation}: Wahrung des finanziellen Gleichgewichts erfordert
\begin{itemize}
	\item Detaillierte Planung zukünftiger Ein- und Auszahlungen, um den Kapitalbedarf rechtzeitig zu identifizieren
	\item Bestimmung der vorzuhaltenden Liquiditätsreserven (Cash Management) und die Messung von Liquidität
	\item Verhindern von absehbaren oder akuten Liquiditätsengpässen (Working Capital Management, Kurzfristfinanzierung)
\end{itemize}
\bigskip
\textbf{Was ist Cash bzw. Liquidität?}
\begin{itemize}
	\item \textbf{Zahlungsmittel}: Kassenbestand, Kredite und Schecks
	\item \textbf{Zahlungsmitteläquivalente}: Kurzfristige, sehr liquide Geldanlagen wie z.B. Schatzbriefe oder Geldmarktfonds $\rightarrow$ leicht veräußerbar, geringe Wertänderungsrisiken
\end{itemize}
\bigskip
\textbf{Motive und Determinanten der Liquiditätshaltung}:
\begin{itemize}
	\item \textbf{Motive}: Vorsichtsmotiv, strategische Motive, Transaktionsmotive
	\item \textbf{Determinanten}:
	\begin{itemize}
		\item Volatilität der Cash Zu- und Abflüsse $[+]$
		\item Kapitalmarktzugang und Kreditfähigkeit des Unternehmens $[-]$
		\item Effizienz des Cash-Flow bzw. Working Capital Management $[-]$
	\end{itemize}
\end{itemize}
\bigskip
\textbf{Kosten der Liquiditätshaltung}: $\rightarrow$ \textbf{Opportunitätskosten}, z.B. Entgangene Zinserträge, Steuernachteile

\textbf{Kosten unzureichender Liquiditätsreserven}: $\rightarrow$ \textbf{Transaktionskosten}, Kosten für kurzfristige Kreditaufnahme 

\begin{center}
	\includegraphics[width=0.5\textwidth]{images/cash-level.png}
\end{center}
\bigskip
\textbf{Liquiditätsgrade}: Möglichkeit, Vermögensgegenstände in Geld umzuwandeln $\rightarrow$ signalisieren kurzfristigen Kreditgebern Zahlungssicherheit
\begin{itemize}
	\item \textbf{Cash Ratio} $=\cfrac{\text{liquide Mittel}}{\text{kurzfristige Verbindlichkeiten}}$ 
	
	gibt an, inwieweit ein Unternehmen seine Zahlungsverpflichtungen durch seine liquiden Mittel erfüllen kann
	
	\item \textbf{Acid Test Ratio}$=\cfrac{\text{liquide Mittel}+\text{kurzfristige Forderungen}}{\text{kurzfristige Verbindlichkeiten}}$
	
	$\text{ATR} < 1$: Ein Teil der kurzfristigen Verbindlichkeiten wird nicht durch kurzfristig zur Verfügung stehendes Vermögen gedeckt
	
	\item \textbf{Current Ratio}$=\cfrac{\text{Umlaufvermögen}}{\text{kurzfristige Verbindlichkeiten}}$
	
	Wert $>$ 1 als Untergrenze, sonst muss Deckung kurzfristiger Verbindlichkeiten durch den Verkauf von Anlagevermögen erfolgen
\end{itemize}
\bigskip
\textbf{Working Capital Management}: Aus der Kapitalbindung im Produktionsprozess resultiert ein bestimmter Kapitalbedarf $\rightarrow$ Managen des Kapitalbedarfs, um Gesamtkosten zu minimieren
\begin{itemize}
	\item \textbf{Working Capital}: Vermögensteile, die sich innerhalb eines Produktionszyklus in liquide Mittel zurückverwandeln
	\item \textbf{Net Working Capital} ist das Nettoumlaufvermögen:
	
	$\text{NWC}=(\text{Umlaufvermögen}-\text{liquide Mittel}-\text{kurzfr. finanz. Vermögenswerte})-(\text{kurzfr. Verbindlichkeiten}-\text{kurzfr. Finanzverbindlichkeiten})$
	
	\item \textbf{Hauptbestandteile des Net Working Capital}:
	\begin{center}
		\includegraphics[width=0.75\textwidth]{images/nwc.png}
	\end{center}
\end{itemize}

\textbf{Cash Conversion Cycle (CCC)}:
\begin{center}
	\includegraphics[width=0.6\textwidth]{images/ccc.png}
\end{center}
\begin{itemize}
	\item Länge des CCC bestimmt den Bedarf an Net Working Capital und damit auch Finanzierungsbedarf und Finanzierungskosten
	\item \textbf{Ziel}: \textbf{Geldumschlagsdauer} (Kapitalbindung) gering halten
	
	$\text{Geldumschlagsdauer}=\text{Durchschnittliche Lagerdauer } + \text{ Durchnittliche Inkassoperiode} \newline \text{(Debitorenlaufzeit)}-\text{Lieferantenzahlungsziel}$
	
	mit $\text{Durchschnittliche Lagedauer}=\cfrac{\text{Durchschn. Lagerbestand} \cdot 360 \text{ Tage}}{\text{Jahresverbrauch}}$
\end{itemize}
\bigskip
\textbf{Ziel des Working Capital Management}: Reduzierung des Net Working Capital und somit Reduktion der Finanzierungskosten

\textbf{Maßnahmen des Working Capital Management}:
\begin{enumerate}
	\item Management der Vorratshaltung und Produktion
	\item Forderungsmanagement: 
	\begin{itemize}
		\item \textbf{Handelskredite}: Unternehmen nehmen Kredite von Lieferanten auf und gewähren ihren Kunden Kredite
		\item \textbf{Factoring}: Verkauf von Forderungen an eine Spezialbank (Factor), Unternehmen und Factor einigen sich auf Konditionen
		\begin{center}
			\includegraphics[width=0.6\textwidth]{images/factoring.png}
		\end{center}
		\pagebreak
		
		\item \textbf{Reverse Factoring}: 
		\begin{center}
			\includegraphics[width=0.6\textwidth]{images/reverse-factoring.png}
		\end{center}
	\end{itemize}
	\item Management der Verbindlichkeiten
\end{enumerate}
\bigskip
\textbf{Politiken der Kurzfristfinanzierung}:
\begin{itemize}
	\item Deckung langfristiger Investitionen durch Langfristfinanzierung
	\item Deckung kurzfristiger Investitionen durch Kurzfristfinanzierung
	\item Finanzierung von langfr. Betriebskapital mit kurzfristigem Kapital: \textbf{aggressive Finanzierungspolitik} $\rightarrow$ führt zu hohem Finanzierungsbedarf $\rightarrow$ Opportunitätskosten
	\item Finanzierung von kurzfr. Betriebskapital mit langfristigem Kapital: \textbf{konservative Finanzierungspolitik} $\rightarrow$ führt zu niedrigerem Finanzierungsbedarf $\rightarrow$ potentieller Verlust von Kunden, Finanzierungsengpässe
\end{itemize}

