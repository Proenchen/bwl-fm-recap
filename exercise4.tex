\section{Übung 4 - Kapitalstruktur und Kapitalkosten}

\subsection{Modigliani-Miller – Perfekte Kapitalmärkte}

\textbf{Theorem 1}: \textit{VG steht für Verschuldungsgrad}
\begin{center}
	\includegraphics[width=0.7\textwidth]{images/e7.png}
\end{center}
$\rightarrow$ Irrelevanz der Kapitalstruktur

\textbf{Theorem 2}: 
$$r_{WACC}=r_U=r_A=\frac{E}{E+D}\cdot r_E+\frac{D}{E+D}\cdot r_D$$
wobei $r_X$ für die Rendite von $X$ steht.

\underline{Achtung}: Bei Steuern muss noch der Term $(1-T)$ hinzugefügt werden!

\textbf{Theorem 3}:
$$r_E=r_U+\frac{D}{E}(r_U-r_D)$$
wobei der letzte Summand für den Risikoaufschlag auf EK-Rendite durch FK-Finanzierung steht\\

\textbf{Einflussfaktoren auf den optimalen Verschuldungsgrad}:
\begin{itemize}
	\item Stabilität / Höhe der Cash Flows (Profitabilität) (+)
	\item Unternehmensgröße (+)
	\item Wachstumsmöglichkeiten (-)
	\item Starke Wettbewerbsposition / Branchenstruktur (+)
\end{itemize}
(+), (-) gibt an, ob Unternehmen mit einer starken Ausprägung der Eigenschaft viel oder wenig FK nutzen

\subsection{Wichtige Agency-Kosten}
\textbf{Übermäßige Risikobereitschaft und Asset Substitution}:\\
Im Falle einer finanziellen Notlage können EK-Geber profitieren,  das Unternehmensrisiko zu erhöhen, auch wenn diese einen negativen Barwert aufweisen $\rightarrow$ Ersetzung von wenig riskanten Vermögensgegenstände durch riskantere (\textbf{Asset Substitution})

\textbf{Debt Overhang und Unterinvestition}:\\
EK-Geber sind nicht gewillt in finanzieller Notlage neue wertschaffenden Projekte zu finanzieren, weil ihnen durch die Fremdkapitalfinanzierung nur ein Teil der Erträge wieder zufließen $\rightarrow$ Unterfinanzierung

\textbf{Cashing Out}:\\
Im einer finanziellen Notlage haben EK-Geber Anreiz so viel Liquidität wie möglich aus dem Unternehmen abzuziehen (z.B. Sonderdividende) $\rightarrow$ Bei Insolvenz erhalten vorrangig nur FK-Geber etwas

\subsection{Kapitalkosten}
\textbf{Erweiterte Formel für WACC}:
\begin{center}
	\includegraphics[width=0.9\textwidth]{images/e8.png}
\end{center}
$k_X$: Kosten für $X$, $w_X$: Gewicht von $X$, VZ: Vorzugsaktien, kfr. BK: kurzfristiger Bankkredit