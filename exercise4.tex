\section{Übung 4 - Kapitalstruktur und Kapitalkosten}

\subsection{Modigliani-Miller – Perfekte Kapitalmärkte}

\textbf{Theorem 1}: \textit{VG steht für Verschuldungsgrad}
\begin{center}
	\includegraphics[width=0.7\textwidth]{images/e7.png}
\end{center}
$\rightarrow$ Irrelevanz der Kapitalstruktur

\textbf{Theorem 2}: 
$$r_{WACC}=r_U=r_A=\frac{E}{E+D}\cdot r_E+\frac{D}{E+D}\cdot r_D$$
wobei $r_X$ für die Rendite von $X$ steht.

\underline{Achtung}: Bei Steuern muss noch der Term $(1-T)$ hinzugefügt werden!

\textbf{Theorem 3}:
$$r_E=r_U+\frac{D}{E}(r_U-r_D)$$
wobei der letzte Summand für den Risikoaufschlag auf EK-Rendite durch FK-Finanzierung steht
