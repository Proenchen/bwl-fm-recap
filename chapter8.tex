\section{Aspekte der Investition/Desinvestition und Diversifikation}
\begin{itemize}
	\item Um zu diversifizieren, müssen Unternehmen Investitionen tätigen
	\item Um zu refokussieren, müssen Unternehmen Desinvestitionen tätigen, d.h. Verkauf von Unternehmensteilen (\textbf{Asset Sales}), Verschlanken von Produktlinien
	\item Desinvestitionen können gewollt oder von Regulierungsbehörden auferlegt sein oder es kann zu Notverkäufen kommen, z.B. wegen drohender Insolvenz
\end{itemize}
\bigskip
\textbf{Asset Sales}:
\begin{itemize}
	\item Bei der Veräußerung fließt dem Unternehmen ein großer Kapitalbetrag zu, der überwiegend aus einer Cash-Komponente besteht
	\item Es kommt zu \textbf{Agency-Problemen}, da die zufließenden Mittel unter der Kontrolle des Managements sind
	$\rightarrow$ Grund für den Asset Sale sowie Investitionsmöglichkeiten des Unternehmens sind für die Kapitalgeber von großer Bedeutung
	\item Asset Sales und Asset Purchases dienen dazu, die optimale Unternehmensgröße zu erreichen
\end{itemize}

\textbf{Carve-outs und Spin-offs}:
\begin{itemize}
	\item Veräußerung von Konzernunternehmen bzw. -teilen mit eigener Börsennotierung
	\item \textbf{Vorteile}: 
	\begin{itemize}
		\item Bessere Anreize für das Management, da Börsennotierung eine Beteiligung am EK und Aktienoptionspläne ermöglicht
		\item Höhere Transparenz, da das ausgegliederte Unternehmen ein separates Zahlenwerk liefert
	\end{itemize}
	\item \textbf{Spin-off}: Anteile am auszugliedernden Unternehmen werden an die Konzernaktionäre ausgegeben $\rightarrow$ Kein Mittelzufluss, kein unmittelbarer Eigentümerwechsel
	\item \textbf{Carve-out}: 
	\begin{itemize}
		\item Anteile am auszugliedernden Unternehmen werden  am Kapitalmarkt angeboten (Kapitalmehrheit verbleibt oft beim Mutterkonzern)
		\item Mittelzufluss für die Mutter, außer neue Aktien stammen aus Kapitalerhöhung
		\item Mittelzufluss für die Tochter bei Kapitalerhöhung
	\end{itemize}
\end{itemize}
\bigskip
\textbf{Diversifikation}:



