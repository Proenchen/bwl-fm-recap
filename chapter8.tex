\section{Aspekte der Investition/Desinvestition und Diversifikation}
\begin{itemize}
	\item Um zu diversifizieren, müssen Unternehmen Investitionen tätigen
	\item Um zu refokussieren, müssen Unternehmen Desinvestitionen tätigen, d.h. Verkauf von Unternehmensteilen (\textbf{Asset Sales}), Verschlanken von Produktlinien
	\item Desinvestitionen können gewollt oder von Regulierungsbehörden auferlegt sein oder es kann zu Notverkäufen kommen, z.B. wegen drohender Insolvenz
\end{itemize}
\bigskip
\textbf{Asset Sales}:
\begin{itemize}
	\item Bei der Veräußerung fließt dem Unternehmen ein großer Kapitalbetrag zu, der überwiegend aus einer Cash-Komponente besteht
	\item Es kommt zu \textbf{Agency-Problemen}, da die zufließenden Mittel unter der Kontrolle des Managements sind
	$\rightarrow$ Grund für den Asset Sale sowie Investitionsmöglichkeiten des Unternehmens sind für die Kapitalgeber von großer Bedeutung
	\item Asset Sales und Asset Purchases dienen dazu, die optimale Unternehmensgröße zu erreichen
\end{itemize}

\textbf{Carve-outs und Spin-offs}:
\begin{itemize}
	\item Veräußerung von Konzernunternehmen bzw. -teilen mit eigener Börsennotierung
	\item \textbf{Vorteile}: 
	\begin{itemize}
		\item Bessere Anreize für das Management, da Börsennotierung eine Beteiligung am EK und Aktienoptionspläne ermöglicht
		\item Höhere Transparenz, da das ausgegliederte Unternehmen ein separates Zahlenwerk liefert
	\end{itemize}
	\item \textbf{Spin-off}: Anteile am auszugliedernden Unternehmen werden an die Konzernaktionäre ausgegeben $\rightarrow$ Kein Mittelzufluss, kein unmittelbarer Eigentümerwechsel
	\item \textbf{Carve-out}: 
	\begin{itemize}
		\item Anteile am auszugliedernden Unternehmen werden  am Kapitalmarkt angeboten (Kapitalmehrheit verbleibt oft beim Mutterkonzern)
		\item Mittelzufluss für die Mutter, außer neue Aktien stammen aus Kapitalerhöhung
		\item Mittelzufluss für die Tochter bei Kapitalerhöhung
	\end{itemize}
\end{itemize}
\bigskip
\textbf{Diversifikation}:
\begin{itemize}
	\item Messung des Diversifikationsgrades mithilfe des \textbf{Berry-Index}:
	$$D_B=1-\sum\limits_{i=1}^n p_i^2$$
	mit $D_B$: Diversifikationsmaß nach Berry, $n$: Anzahl der Segmente des Unternehmens, $p_i$: Umsatzanteil $i$ am Gesamtumsatz
	\item \textbf{Interpretation}: 
	\begin{itemize}
		\item $D_B = 0$: Unternehmen ist nur in einem Bereich tätig, d.h. nicht diversifiziert
		\item Je größer $D_B$, desto stärker ist das Unternehmen diversifiziert
	\end{itemize}
\end{itemize}

\textbf{Historische Entwicklung}: 
\begin{itemize}
	\item 1950-1970: Bildung von Konglomeraten, also Unternehmen, das viele Tochterunternehmen hat und in vielen Branchen tätig ist
	\item 1980 - heute: Aufspaltung der Konglomerate $\rightarrow$ Stärkere Fokussierung von Unternehmen
	\item 1988 - 2002: Anstieg der Refokussierung (De-Diversifikationen) von Unternehmen in Deutschland
	\item \textbf{Gründe}: Verwässerung von Kernkompetenzen, Kontrollverluste, Steigende Komplexität, Agency-Probleme
\end{itemize}
\bigskip
\textbf{Gründe für Diversifikation}: 
\begin{itemize}
	\item \textbf{Aus Investorensicht}: Diversifikation kann der Investor selbst in seinem Portfolio erreichen $\rightarrow$ Investoren kümmern sich nicht um die Diversifikation eines Unternehmens
	\item \textbf{Unternehmenswachstum}: Wachstum in einer Branche ist beschränkt $\rightarrow$ Diversifikation als Mittel, um Unternehmens-Wachstum zu erreichen
	\item \textbf{Aus Sicht des Managements}: Manager können ihr nicht diversifizierbares \textbf{Employment Ris}k (Risiko des Jobverlusts und Risiko von Reputationsschäden)
	reduzieren, da Diversifikation das Insolvenzrisiko von Unternehmen reduziert
\end{itemize}

\textbf{Gründe: Interne Kapitalmärkte}
\begin{itemize}
	\item \textbf{Idee}: Zusammenführung eines Unternehmens, das profitable Investitionsprojekte hat, jedoch unter Finanzierungsrestriktionen leidet, mit einem Unternehmen, das über hohe Cash Flows aber unzureichende Investitionsmöglichkeiten verfügt $\rightarrow$ Zusammenschluss kann wertsteigernd sein
	\item \textbf{Merkmale von internen Kapitalmärkten}:
	\begin{itemize}
		\item Unternehmensbereiche unterstehen dem Headquarter $\rightarrow$ Erhalten von dort Mittel zugewiesen
		\item Headquarter hat Eigentümerrechte an den Kapitalerträgen der einzelnen Bereiche $\rightarrow$ Möglichkeit jederzeit Kapital abzuziehen
	\end{itemize}
	\item \textbf{Vorteile von internen Kapitalmärkten}:
	\begin{itemize}
		\item \textbf{\enquote{More money} Effekt}: Diversifikation reduziert die Insolvenzwahrscheinlichkeit 
		\item \textbf{\enquote{Smart money} Effekt}: Effizientere Kapitalallokation durch Nutzung interner Informationen über Investitionsprojekte
		\item Bessere Geheimhaltung von Investitionsideen und schnellere Reaktionsmöglichkeiten durch Innenfinanzierung
	\end{itemize}
	\item \textbf{Nachteile interner Kapitalmärkte}:
	\begin{itemize}
		\item Nicht alle Entscheidungen führen zu effizienter Kapitalallokation
		\item Demotivation der Bereichsmanager durch Mittelabzug bei guten Ergebnissen
		\item Verhandlungsposition der Bereichsleiter haben Einfluss auf Kapitalallokation
		\item Erfordert Know-How des Managements in allen vertretenen Geschäftsfeldern
	\end{itemize}
\end{itemize}

\textbf{Gründe: Verminderung von Risiken}
\begin{itemize}
	\item \textbf{Unsystematisches Risiko}, z.B. saisonale Nachfrageschwankungen, kann durch Diversifikation verringert werden
	\item Systematisches Risiko kann durch Diversifikation normalerweise nicht ausgeschaltet werden
	\item Aktuelle Studien zeigen aber: Diversifizierte Unternehmen können Kapital- und Insolvenzkosten reduzieren $\rightarrow$ Positiver Effekt auf systematisches Risiko (\textbf{Co-Insurance Effekt})
\end{itemize}
\bigskip
\textbf{Conglomerate Discount}:
\begin{itemize}
	\item Sollten Unternehmen diversifizieren? $\rightarrow$ Ja, wenn es den Unternehmenswert steigert
	\item \textbf{Studie von Berger und Ofek} zeigt aber mit Bewertung von
	\begin{itemize}
		\item Marktwert / Bilanzsumme
		\item Marktwert / Umsatz
		\item Marktwert / EBIT
	\end{itemize}
	dass diversifiziertes Unternehmen einen geringeren Marktwert als ein \enquote{Vergleichsportfolio} spezialisierter Unternehmen hat! (\textbf{Conglomerate Discount})
	\item Neue Studien stellen das aber in Frage $\rightarrow$ Existenz eines Conglomerate Discount bis jetzt nicht endgültig geklärt 
\end{itemize}